
%------------------------------------------------------------------------------
\chapter{浮点数加法次序的问题}

%------------------------------------------------------------------------------
\section{实验目标}

\begin{enumerate}
	\item 学习了解浮点数运算次序引起的问题;
	\item 学习了解浮点数运算产生误差的原因。
\end{enumerate}

%------------------------------------------------------------------------------
\section{实验条件}

\begin{enumerate}
	\item Linux操作系统(推荐),或Windows操作系统;
	\item 程序开发工具GCC(Linux),或DevC++(Windows)。
\end{enumerate}

%------------------------------------------------------------------------------
\section{实验准备}

\begin{enumerate}
	\item 回顾教材第三章关于浮点数表示以及浮点数运算尤其是加减法运算的过程;
	\item 利用图书馆或互联网资源,了解浮点运算产生误差的原因和需要注意的问题\\
		(例如: https://docs.oracle.com/cd/E19957-01/806-3568/ncg\_goldberg.html
		
		以及 https://en.wikipedia.org/wiki/Floating-point\_arithmetic\#
		
		Causes\_of\_Floating\_Point\_Error )。
\end{enumerate}

%------------------------------------------------------------------------------
\section{实验内容}

\begin{enumerate}
	\item 从实验说明文档包中找到 floatsum.cpp 程序,编译并运行;
		%\lstinputlisting[language=c]{floatsum.cpp}
	\item 测试3个不同种子数(1, 3, 5)和3个不同的数组大小(30, 40, 50)的组合,观察数组累加结果并截图记录, 以便后续分析。
\end{enumerate}

%------------------------------------------------------------------------------
\section{思考问题}

\begin{enumerate}
	\item 浮点数运算产生误差的根本原因是什么?
	\item 为什么浮点加法运算的次序会影响误差的大小?
	\item 怎么样减小浮点运算的误差?
\end{enumerate}

%------------------------------------------------------------------------------
\section{报告要求}

\begin{enumerate}
	\item 实验报告需要有姓名,学号,班级,实验名称,实验目标,实验条件,实验内容,实验记录,实验分析等项目;
	\item 实验记录需要有如下信息:(1)实验机器CPU配置,内存配置,操作系统名称和版本号,GCC或DevC++版本号;(2)程序源代码,以及GCC编译命令文本;(3)测试3个不同种子数(1, 3, 5)和3个不同的数组大小(30, 40, 50)组合时程序的输出(请截图)。
	\item 实验分析要根据实验记录得到的结果,回答思考问题中各个问题。
\end{enumerate}

