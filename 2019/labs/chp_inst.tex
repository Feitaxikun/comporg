
%------------------------------------------------------------------------------
\chapter{指令的表示}

%------------------------------------------------------------------------------
\section{实验目标}

\begin{enumerate}
	\item 学习了解C/C++语言程序控制结构对应的指令序列;
	\item 学习调试器的使用和查看可执行程序的汇编代码。
\end{enumerate}

%------------------------------------------------------------------------------
\section{实验条件}

\begin{enumerate}
	\item Linux操作系统(推荐),或Windows操作系统;
	\item 程序开发工具GCC(Linux),或DevC++(Windows)。
\end{enumerate}

%------------------------------------------------------------------------------
\section{实验准备}

\begin{enumerate}
	\item 回顾教材第二章中关于C/C++语言中条件分支和循环结构对应的指令序列;
	\item 利用图书馆或互联网资源,了解调试器用法(例如: 

		https://www.gdb-tutorial.net/ )。
\end{enumerate}

%------------------------------------------------------------------------------
\section{实验内容}

\begin{enumerate}
	\item 编写一个程序,包含下述代码:
\begin{lstlisting}[language={C++}]
int a;

while(1) {

    printf("\n输入一个整数(负数将退出程序):");
    scanf("%d", &a);

    if(a < 0) break;
}
\end{lstlisting}

	编译并改正可能的错误,使得程序运行正确;

	\item 利用调试器查看程序的汇编代码,找到对应 while 循环语句和 if 条件分支语句的汇编代码段,并做记录。
\end{enumerate}

%------------------------------------------------------------------------------
\section{思考问题}

\begin{enumerate}
	\item C/C++语言中 while 循环语句通常被编译成什么样的指令序列?
	\item C/C++语言中 if 条件分支语句通常被编译成什么样的指令序列?
	\item 实验中的无限 while 循环和其中的 if 分支语句被编译成了哪些指令?
\end{enumerate}

%------------------------------------------------------------------------------
\section{报告要求}

\begin{enumerate}
	\item 实验报告需要有姓名,学号,班级,实验名称,实验目标,实验条件,实验内容,实验记录,实验分析等项目;
	\item 实验记录需要有如下信息:(1)实验机器CPU配置,内存配置,操作系统名称和版本号,GCC或DevC++版本号;(2)程序源代码,GCC编译命令文本, 利用调试器查看汇编代码的命令或截图;(3)查看到的程序汇编代码截图。
	\item 实验分析要根据实验记录得到的结果,回答思考问题中各个问题。
\end{enumerate}

