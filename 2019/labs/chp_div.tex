
%------------------------------------------------------------------------------
\chapter{用加减法实现二进制整数除法}

%------------------------------------------------------------------------------
\section{实验目标}

\begin{enumerate}
	\item 学习了解二进制整数除法的运算方法;
	\item 学习了解二进制整数除法中对有符号数的处理方法。
\end{enumerate}

%------------------------------------------------------------------------------
\section{实验条件}

\begin{enumerate}
	\item Linux操作系统(推荐),或Windows操作系统;
	\item 程序开发工具GCC(Linux),或DevC++(Windows)。
\end{enumerate}

%------------------------------------------------------------------------------
\section{实验准备}

\begin{enumerate}
	\item 回顾教材第三章关于二进制整数除法运算的过程;
	\item 利用图书馆或互联网资源, 了解二进制整数除法的运算方法\\
		(例如: https://en.wikipedia.org/wiki/Division\_algorithm)。
\end{enumerate}

%------------------------------------------------------------------------------
\section{实验内容}

\begin{enumerate}
	\item 从实验说明文档包中找到 div.cpp 程序,编译并运行;
		%\lstinputlisting[language=c]{div.cpp}
	\item 观察 div.cpp 程序的运行错误,改正其错误并再次编译运行;
	\item 测试除数、被除数分别是正数、负数和零的情况,记录测试结果(截图)。
\end{enumerate}

%------------------------------------------------------------------------------
\section{思考问题}

\begin{enumerate}
	\item div.cpp 采用的二进制整数除法是哪一种?
	\item div.cpp 的错误在什么地方,如何改正?
	\item div.cpp 中对商和余数的符号是如何处理的?
\end{enumerate}

%------------------------------------------------------------------------------
\section{报告要求}

\begin{enumerate}
	\item 实验报告需要有姓名,学号,班级,实验名称,实验目标,实验条件,实验内容,实验记录,实验分析等项目;
	\item 实验记录需要有如下信息:(1)实验机器CPU配置,内存配置,操作系统名称和版本号,GCC或DevC++版本号;(2)改正后的程序源代码,以及GCC编译命令文本;(3)测试除数、被除数分别是正数、负数和零的情况,记录测试结果(请截图)。
	\item 实验分析要根据实验记录得到的结果,回答思考问题中各个问题。
\end{enumerate}

