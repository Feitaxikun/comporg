
%------------------------------------------------------------------------------
\chapter{测量程序运行时间}

%------------------------------------------------------------------------------
\section{实验目标}

\begin{enumerate}
	\item 学习使用clock()函数记录程序执行的CPU时间;
	\item 学习使用gettimeofday()函数记录实际时间。
\end{enumerate}

%------------------------------------------------------------------------------
\section{实验条件}

\begin{enumerate}
	\item Linux操作系统(推荐),或Windows操作系统;
	\item 程序开发工具GCC(Linux),或DevC++(Windows)。
\end{enumerate}

%------------------------------------------------------------------------------
\section{实验准备}

\begin{enumerate}
	\item 安装Linux操作系统,或Windows操作系统;
	\item 如果Linux或Windows没有自带程序开发工具,那么需要安装GCC(Linux)或DevC++(Windows);
	\item 利用图书馆或互联网资源,了解标准C/C++运行时库的clock()和gettimeofday()函数用法(例如: http://rabbit.eng.miami.edu/info/ \\
		functions/time.html )。
\end{enumerate}

%------------------------------------------------------------------------------
\section{实验内容}

\begin{enumerate}
	\item 编写一个矩阵相乘的C程序,矩阵大小n根据实验机器性能自行设置,建议为2000。相乘部分的参考代码如下所示:
\begin{lstlisting}[language={C++}]
for (k=0; k<n; k++) {
    for (i=0; i<n; i++) {
    	r = a[i][k];
    	for (j=0; j<n; j++) {
    		c[i][j] += r * b[k][j];
    	}
    }
}
\end{lstlisting}
编译运行后,检查程序运行是否正确;
	\item 用clock()和gettimeofday()函数获取矩阵相乘程序运行时间(包括CPU时间和实际时间),建议重复30次并计算平均值作为结果。
\end{enumerate}

%------------------------------------------------------------------------------
\section{思考问题}

\begin{enumerate}
	\item 程序运行的CPU时间是什么意思?CPU时间和实际时间有什么联系和区别;
	\item clock()函数获取的CPU时间精度是多少?
	\item gettimeofday()函数获取的实际时间是什么含义,精度是多少?
\end{enumerate}

%------------------------------------------------------------------------------
\section{报告要求}

\begin{enumerate}
	\item 实验报告需要有姓名,学号,班级,实验名称,实验目标,实验条件,实验内容,实验记录,实验分析等项目;
	\item 实验记录需要有如下信息:(1)实验机器CPU配置,内存配置,操作系统名称和版本号,内核版本号,GCC版本号;(2)矩阵相乘程序源代码,GCC编译命令文本;(3)clock()和gettimeofday()函数的使用要有程序源代码,实验中设置的具体矩阵大小,重复次数,每次测试的输出结果,计算得到的平均CPU时间,平均实际时间。
	\item 实验分析要根据实验记录得到的结果,回答思考问题中各个问题。
\end{enumerate}

