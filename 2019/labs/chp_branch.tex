
%------------------------------------------------------------------------------
\chapter{分支预测准确率对程序执行时间的影响}

本实验中示例程序采用了通信1603班盛晓甜同学在讨论课上所演示程序,特此鸣谢!

%------------------------------------------------------------------------------
\section{实验目标}

\begin{enumerate}
	\item 学习了解分支预测技术的工作原理;
	\item 学习了解分支预测准确率对程序性能的影响。
\end{enumerate}

%------------------------------------------------------------------------------
\section{实验条件}

\begin{enumerate}
	\item Linux操作系统(推荐),或Windows操作系统;
	\item 程序开发工具GCC(Linux),或DevC++(Windows)。
\end{enumerate}

%------------------------------------------------------------------------------
\section{实验准备}

\begin{enumerate}
	\item 回顾教材第四章关于分支预测的内容;
	\item 利用图书馆或互联网资源,了解分支预测对程序性能的影响(例如: 

		http://www.cnblogs.com/yangecnu/p/4196026.html )。
\end{enumerate}

%------------------------------------------------------------------------------
\section{实验内容}

\begin{enumerate}
	\item 从实验说明文档包中找到 branchprediction\_random.cpp 和 
		branchprediction\_sorted.cpp 程序,编译并运行;
	\item 观察程序执行时间并截图记录, 以便后续分析。
\end{enumerate}

%------------------------------------------------------------------------------
\section{思考问题}

\begin{enumerate}
	\item 分支预测是为了解决流水线中什么问题而提出来的?分支预测是如何实现的?
	\item 分支预测成功和失败时,流水线中的情况分别是什么样的?
	\item 示例程序中未排序和排序数组对分支预测的影响是什么?
\end{enumerate}

%------------------------------------------------------------------------------
\section{报告要求}

\begin{enumerate}
	\item 实验报告需要有姓名,学号,班级,实验名称,实验目标,实验条件,实验内容,实验记录,实验分析等项目;
	\item 实验记录需要有如下信息:(1)实验机器CPU配置,内存配置,操作系统名称和版本号,GCC或DevC++版本号;(2)程序源代码,以及GCC编译命令文本;(3)示例程序的输出(请截图)。
	\item 实验分析要根据实验记录得到的结果,回答思考问题中各个问题。
\end{enumerate}

